\sloppy

% Настройки стиля ГОСТ 7-32
% Для начала определяем, хотим мы или нет, чтобы рисунки и таблицы нумеровались в пределах раздела, или нам нужна сквозная нумерация.
\EqInChapter % формулы будут нумероваться в пределах раздела
\TableInChapter % таблицы будут нумероваться в пределах раздела
\PicInChapter % рисунки будут нумероваться в пределах раздела

% Добавляем гипертекстовое оглавление в PDF
\usepackage[
bookmarks=true, colorlinks=true, unicode=true,
urlcolor=black,linkcolor=black, anchorcolor=black,
citecolor=black, menucolor=black, filecolor=black,
]{hyperref}

% Изменение начертания шрифта --- после чего выглядит таймсоподобно.
% apt-get install scalable-cyrfonts-tex

\IfFileExists{cyrtimes.sty}
	{
		\usepackage{cyrtimespatched}
	}
	{
		% А если Times нету, то будет CM...
	}

\usepackage{graphicx}   % Пакет для включения рисунков
\DeclareGraphicsExtensions{.jpg,.pdf,.png}
% С такими оно полями оно работает по-умолчанию:
% \RequirePackage[left=20mm,right=10mm,top=20mm,bottom=20mm,headsep=0pt]{geometry}
% Если вас тошнит от поля в 10мм --- увеличивайте до 20-ти, ну и про переплёт не забывайте:
\geometry{right=20mm}
\geometry{left=30mm}



% Произвольная нумерация списков.
\usepackage{enumerate}

\setcounter{tocdepth}{1} %Подробность оглавления
%4 это chapter, section, subsection, subsubsection и paragraph
%3 это chapter, section, subsection и subsubsection
%2 это chapter, section, и subsection
%1 это chapter и section
%0 это chapter.
\usepackage{mathptmx} % Times New Roman в формулах
\usepackage{cyrtimespatched} % Times New Roman в тексте
\usepackage[defaultmono]{droidmono} % Droid Mono для моноширинного текста

\usepackage{booktabs} % Пакет с таблицами, как в научных публикациях

\usepackage{listings} % Вставка исходных кодов программ
\usepackage{color} % Объявление цветов для ключевых слов
\usepackage{textcomp}

\usepackage{multirow} % Объединение строк в таблицах
\usepackage{multicol} % Объединение столбцов в таблицах

\usepackage{chemfig} % Вставка химических формул и реакций
\usepackage{siunitx} % Единицы измерения по СИ

\usepackage{url} % Вставка URL адресов, например, в библиографию
\usepackage{subcaption} % Несколько изображений со своими подписями, объединенные одной общей

% ----------------- coding settings ------------------
\definecolor{listinggray}{gray}{0.5}
\definecolor{lbcolor}{rgb}{1,1,1}
\renewcommand*\lstlistingname{Листинг}
\lstset{
	backgroundcolor=\color{lbcolor},
	tabsize=4,
	rulecolor=,
	% language=set your base language,
	basicstyle=\footnotesize\fontencoding{T1}\ttfamily,
	upquote=true,
	aboveskip={1\baselineskip},
	columns=fixed,
	showstringspaces=false,
	extendedchars=true,
	breaklines=true,
	mathescape=true,
	prebreak = \raisebox{0ex}[0ex][0ex]{\ensuremath{\hookleftarrow}},
	frame=single,
	showtabs=false,
	showspaces=false,
	showstringspaces=false,
	keywordstyle=\color[rgb]{0, 0, 1},
	commentstyle=\color[rgb]{0.133, 0.545, 0.133},
	stringstyle=\color[rgb]{0.627, 0.126, 0.941},
}
% ----------------------- end ------------------------

% ----------------- chemistry settings ------------------
\newcommand\setpolymerdelim[2]{\def\delimleft{#1}\def\delimright{#2}}
\def\makebraces[#1,#2]#3#4#5{%
\edef\delimhalfdim{\the\dimexpr(#1+#2)/2}%
\edef\delimvshift{\the\dimexpr(#1-#2)/2}%
\chemmove{%
\node[at=(#4),yshift=(\delimvshift)]
{$\left\delimleft\vrule height\delimhalfdim depth\delimhalfdim
width0pt\right.$};%
\node[at=(#5),yshift=(\delimvshift)]
{$\left.\vrule height\delimhalfdim depth\delimhalfdim
width0pt\right\delimright_{\rlap{$\scriptstyle#3$}}$};}}
\setpolymerdelim()
\setatomsep{2em}
% ------------------------- end -------------------------
